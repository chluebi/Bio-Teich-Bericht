\documentclass{article}
\usepackage[T1]{fontenc}
\usepackage[utf8]{inputenc}



\usepackage{graphicx}
\usepackage{sectsty}
\usepackage{comment}
\usepackage{lmodern}
\usepackage{textcomp}
\usepackage{chemformula}
% \usepackage[ansinew]{inputenc}

\usepackage[ngerman]{babel}


\usepackage{geometry}
 \geometry{
 a4paper,
 total={170mm,257mm},
 left=30mm,
 right=30mm,
 top=20mm,
 }
 
 \title{\Huge{Untersuchung der Rämibühlteiche}}
\author{ \huge{Bas und Luis} \\ \\ \\
         \centering{\includegraphics[width=15cm]{Teichtitelbild.JPG}}}
\date{Mai 2019}

\begin{document}

\maketitle

\newpage


\centering \section{Zusammenfassung}
Es wurde beobachtet, dass die Rämibühl-Teiche sehr sehr sehr sehr sehr sehr sehr sehrsehr sehr sehr sehrsehr sehr sehr sehrsehr sehr sehr sehrsehr sehr sehr sehr interessant sind.

\begin{figure}[h!]
\centering
\includegraphics[scale=0.15]{hydra.png}
\caption{Hydra}
\label{fig:universe}
\end{figure}

\section{Einleitung}

    \subsection{Ökosysteme, Biozönose, Biotop}
        
        Eine \textbf{Biozönose} ist eine allgemeine Lebensgemeinschaft von Produzenten, Konsumenten und Destruenten, die alle jeweils voneinander abhängig sind. \cite{Biobuch}[Seite 317]
        Diese leben alle zusammen in einem \textbf{Biotop}, gängie Beispiele sind Wälder, Seelandschaften, aber auch Wüsten. \\
        \vspace{5mm}
        \textit{
        Ein Ökosystem (griech. oikos = Haus; systema = verbunden) besteht aus dem Verbund von Biotop und Biozönose. Anders ausgedrückt: Der Lebensraum und die darin lebenden Organismen bilden zusammen ein Ökosystem. \cite{Biologie-schule.de} } \\
        
        In einem solchen funktionierenden Verbund findet ein ständiger Kreislauf umgesetzter Stoffe statt \cite{Biobuch} [Seite 317]
        
    \subsection{Teich}
    
       Ein \textbr{Teich} ist das Ökosystem eines kleinen menschengeschaffenen stehenden Gewässers \cite{Duden}. Es ähnelt sehr dem Ökosystem See und viele Dinge können direkt übertragen werden. \cite{Kleingewasserkunde} Dieses zeichnet sich vor allem durch Phytoplankton und Unterwasserpflanzen als Produzenten, wie auch eine riesige Vielfalt an Wassertieren aus. \cite{Biobuch} [Seite 328]
    
    \subsection{stehendes Gewässer}
    
        Stehende Gewässer, auch Stillgewässer genannt sind Gewässer mit keiner oder tiefer Fliessgeschwindigkeit. Sie sind standardgsgemäss in folgende Kategorien unterteilt: \\
        
        - Seen, grosse Gewässer ab 8-10 m Tiefe mit Abfluss und Zufluss \\
        - Weiher, Stillgewässer ohne Abfluss oder Zufluss und (einigermassen konstanter Wassertiefe) \\
        - Tümpel, Stillgewässer ohne Abfluss oder Zufluss, welche periodisch austrocknen \\
        - Teich, menschengeschaffenes Gewässer mit Abfluss und Zufluss und oftmals periodischer Austrocknung \\ \cite{Kleingewasserkunde}
    
    \subsection{Plankton, Nekton}
    
        \textbr{Plankton} ist der Oberbegriff aller im Wasser lebenden Organismen, die sich treiben lassen. Man unterscheidet zwischen Phytoplankton, welcher Fotosynthese betreibt, und Zooplankton, welcher sich von anderen Organismen ernährt. Beispiele sind Cyanobakterien, Wasserflöhe und Mückenlarven \cite{Biobuch} [Seite 328] \\ \\
        \vspace{5mm}
        \textbr{Nekton} hingegen bezeichnet die ``Schwimmwelt``, also alle im Wasser lebende Organismen, welche einen aktiven Ortswechsel durchführen ohne durch die Wasserbewegung behindert zu werden. Dies sind meist grosse Organismen. Beispiele sind Fische, Krebstiere und Reptilien \cite{Spektrum}
    
    \subsection{Einzeller, Mehrzeller}
    
        Man bezeichnet einen Organismus als \textbr{einzellig} wenn dieser aus nur einer einzigen autonomen Zelle besteht. Beispiele sind Euglena und Amöben. \\ \textbr{Mehrzellige} Organismen hingegen basieren auf der und Aufgabenteilung zwischen mehrerer Zellen. Beispiele sind Frösche, Bäume und Trüffel.
        Es ist zu bemerken, dass die Grenze zwischen einzellig und mehrzellig sehr schwammig sein kann: Siehe Volvox.
    
    \subsection{Insekten, Wirbellose, Wirbeltiere}
    
        Als \textbr{Insekten} bezeichnet man traditionell Tiere aus dem Stamm der Gliedfüsser , die unter anderem alle sechs Beine besitzen. Zu bemerken sind unter anderem die kleine Grösse und der Chitinpanzer. Beispiele sind Bienen, Libellen und Fliegen. \cite{EvolutionofInsects} \\
        \vspace{5mm}
        \textbr{Wirbellose} sind eine Formtaxa \cite{UniProtokolle} aller Tiere ohne jegliche Wirbelksäule. Dazu gehören zum Beispiel Insekten, Schnecken und Ringelwürmer. \cite{FrustfreiLernen} \\
        \vspace{5mm}
        \textbr{Wirbeltiere}, auch Schädeltiere genannt,  besitzen im Gegensatz zu den Wirbellosen undeine Wirbelsäule und Schädel. Somit besitzen sie auch einen Rückenmarknerv und einen Körper der in Kopf, Rumpf, Schwanz und zwei Paar Gliedmassen gegliedert werden kann. \cite{Lernhelfer}
        
    \subsection{Rolle der Wasserpflanzen im Teich}
    
        Wasserpflanzen bilden zusammen mit Phytoplankton die unterste Schicht der Nahrungsketten im Teich und somit als Konsumenten überlebenswichtig für ein beständiges Ökosystem \cite{Biobuch} [Seite 328]. Ausserdem produzieren Pflanzen Sauerstoff: \textit{Wasserpflanzen im Teich sind ein Muss – denn sie sind die Grundlage für Leben im Wasser.}  Ausserdem säubern sie das Wasser und unterdrücken das Wachstum von Fadenalgen. \cite{Lupos}
    
    \subsection{Benthal, Litoral}
    
        \begin{figure}[h!]
        \centering
        \includegraphics[scale=0.4]{zonierung.jpg}
        \caption{Zonierung des Wassers, Quelle: \cite{klasseWasser}}
        \label{fig:universe}
        \end{figure}
        
    
        Als \textbr{Benthal} bezeichnet man die oberste Bodenregion im Ökosystem See und bei Teichen, aufgrund ihrer kleinen Tiefe, die gesamte Bodenregion. \cite{Spektrum} Da sich diese Zone unterhalb der Kompensationsschicht befindet, können Pflanzen keine Fotosynthese betreiben. Also finden hier sich Tiere wie Wasserschnecken und andere Wirbellose.
        \vspace{5mm}
        Die ufernahe \textbr{Litoral} dagegen ist oberhalb der Kompensationsschicht. Sie ist stark bewachsen von Wasserpflanzen. Hier wird die gesamte organische Masse aufgebaut von der der Teich lebt.
    
    \subsection{Nahrungskette, Nahrungsnetz}
    
        Als \citebr{Nahrungskette} bezeichnet man eine graphische Darstellung der Nährstoffübergabe in einem Ökosystem.  Man beginnt hierbei mit einem autotrophen Produzenten, welcher Sonnenlicht in Biomasse umwandelt. Die nächste Stufe ist der Primärkonsument. Dieser ernährt sich heterotroph von dieser Biomasse. Darauf folgen Sekundärkonsument und Tertiärkonsument. \cite{Biobuch} [Seite 318] \\
        \vspace{5mm}
        Da jedoch die meisten Lebewesen von vielen verschiedenen Dingen ernähren, benutzt man ein allgemeineres Modell: Ein \citebr{Nahrungsnetz} verbindet alle Lebewesen in einem Ökosystem durch Ernährungsbeziehungen. Man bezeichnet jetzt nur noch allgemeiner als ``Konsumenten`` und ``Produzenten``, da \textit{die Trophiestufen nicht mehr hierarchisch übereinander sondern ineinander verflochten sind.} \cite{Biobuch} [Seite 319]
    
    \subsection{Stoffkreisläufe}
    
        In den meisten Ökosystemen gibt es zwei parallel laufende \textbr{Stoffkreisläufe}: \\
        \vspace{5mm}
        Der organische Stoffkreislauf besteht aus der Beziehung zwischen den Produzenten, Konsumenten und Destruenten. Konsumenten beziehen ihre Energie aus der Biomasse der Produzenten. Konsumeten wie auch Produzenten sterben ab und werden zuerst zu Detritus. Durch die Destruenten werden sie mineralisiert und in ihre anorganischen Stoffteile zersetzt. Diese können wiederum von den Konsumeten aus dem Boden aufgenommen werden. \cite{Biobuch} [Seite 320] \\
        \vspace{5mm}
        Der anorganische Stoffkreislauf besitzt als Herzstück die Sauerstoff-produzierenden Produzenten. Sowohl Destruenten als auch Konsumenten sind von diesem Sauerstoff abhängig. Als Rückgabe produzieren sie Kohlenstoffdioxid, welcher wiederum von den Konsumenten aufgenommen wird.


\section{Material und Methoden}
    
    \subsection{Aufspüren und Analyse der Lebewesen im Teich}
    
        % TODO Lu
    
        \begin{table}[h!]
        \centering
        \begin{tabular}{|c|} 
         \hline
         \\
         1 Zündholzschachtel \\
         1 Filzstift \\[1ex]
         \hline
        \end{tabular}
        \label{Tiere}
        \end{table}
        
    \subsection{Testen der Wasserqualität im Teich}
    
        \begin{table}[h!]
        \centering
        \begin{tabular}{|c|} 
         \hline
         \\
         1 Zündholzschachtel \\
         1 Filzstift \\[1ex]
         \hline
        \end{tabular}
        \label{Praktikum1}
        \end{table}

    

\section{Resultate}

    \subsection{Nahrungsnetzwerke}
        
        % TODO Bas
    
        \subsubsection{Frühling}
        
        \subsubsection{Frühsommer}
        
    \subsection{Gemessene Wasserwerte}
        % TODO Lu


\section{Diskussion}

    \subsection{Analyse der Nahrungsnetzwerke}
    
        % TODO Bas
        
    % \subsection{Planktonanalyse} potenziell
    
    \subsection{Wasseranalyse}
    
        % TODO Lu

\section{Danksagung}



    \begin{thebibliography}{9}
    \centering{
        \bibitem{Biobuch}
       Biologie Heute erweiterte Ausgabe \textit{Verlag: Schroedel Westermann Herausgeber: Dr. Jürgen Braun, Heinrich Joussen, Dr. Andreas Paul, Elsbeth Westendorf-Bröring}
        \bibitem{Biologie-schule.de}
        Website: \textit{http://www.biologie-schule.de/oekosystem.php} \\ 23.5.19
        \bibitem{Duden}
        Website: \textit{https://www.duden.de/rechtschreibung/} \\ 23.5.19
        \bibitem{Kleingewasserkunde}
        Dieter Glandt: Praktische Kleingewässerkunde \textit{Laurenti Verlag, Bielefeld 2006} \\ \textit{Gefunden über Wikipedia}
        \bibitem{Spektrum}
        Website: \textit{https://www.spektrum.de/lexikon/biologie/} \\ 23.5.19
        \bibitem{EvolutionofInsects}
         David Grimaldi, Michael S. Engel: Evolution of the Insects. (= Cambridge Evolution Series).  \textit{Cambridge University Press, 2005} \\ 
         \textit{Gefunden über Wikipedia} \\
         \bibitem{FrustfreiLernen}
        Website: \textit{https://www.frustfrei-lernen.de/biologie/wirbeltiere-wirbellose-tiere-biologie.html} \\  23.5.19
        \bibitem{UniProtokolle}
        Website: \textit{http://www.uni-protokolle.de/Lexikon/Taxon.html} \\ 23.5.19
        \bibitem{Lernhelfer}
        Website: \textit{https://www.lernhelfer.de/schuelerlexikon/biologie/artikel/wirbeltiere} \\ 23.5.19
        \bibitem{Lupos}
        Website: \textit{http://www.lupos-landleben.de/wasserpflanzen-im-teich/} \\ 24.5.19
        \bibitem{klasseWasser}
        Website: \textit{https://www.klassewasser.de/content/language1/html/6148.php} \\ 24.5.19
        
        
        
        
         
    }
    \end{thebibliography}

\end{document}
