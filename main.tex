\documentclass{article}
\usepackage[utf8]{inputenc}

\title{Untersuchung der Rämibühlteiche}
\author{Bas und Luis}
\date{Mai 2019}

\usepackage{natbib}
\usepackage{graphicx}

\renewcommand\bibliography{Literatur}

\begin{document}

\maketitle

\section{Zusammenfassung}
Es wurde beobachtet, dass die Rämibühl-Teiche sehr interessant sind.

\section{Einleitung}

\subsection{Ökosysteme, Biozönose, Biotop}

\subsection{Teich}

\subsection{stehendes Gewässer}

\subsection{Plankton, Nekton}

\subsection{Einzeller, Mehrzeller}

\subsection{Insekten, Wirbelose, Wirbeltiere}

\subsection{Rolle der Wasserpflanzen im Teich}

\subsection{Benthal, Litoral}

\subsection{Nahrungskette, Nahrungsnetz}

\subsection{Stoffkreisläufe}


\section{Material und Methoden}

\section{Resultate}

\section{Diskussion}

\section{Danksagung}

\section{Literatur}

\end{document}
