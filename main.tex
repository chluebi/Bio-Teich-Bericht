\documentclass{article}
\usepackage[utf8]{inputenc}

\title{\Huge{Untersuchung der Rämibühlteiche}}
\author{ \huge{Bas und Luis} \\ \\ \\
         \includegraphics[width=17cm]{Teichtitelbild.JPG}}
\date{Mai 2019}

\usepackage{graphicx}
\usepackage{sectsty}
\usepackage{comment}
\usepackage{lmodern}
\usepackage{textcomp}
\usepackage{chemformula}


\usepackage{geometry}
 \geometry{
 a4paper,
 total={170mm,257mm},
 left=15mm,
 top=15mm,
 }

\begin{document}

\maketitle

\newpage


\centering \section{Zusammenfassung}
Es wurde beobachtet, dass die Rämibühl-Teiche sehr sehr sehr sehr sehr sehr sehr sehrsehr sehr sehr sehrsehr sehr sehr sehrsehr sehr sehr sehrsehr sehr sehr sehr interessant sind.

\begin{figure}[h!]
\centering
\includegraphics[scale=0.15]{hydra.png}
\caption{Hydra}
\label{fig:universe}
\end{figure}

\section{Einleitung}

    \subsection{Ökosysteme, Biozönose, Biotop}
    
    \subsection{Teich}
    
    \subsection{stehendes Gewässer}
    
    \subsection{Plankton, Nekton}
    
    \subsection{Einzeller, Mehrzeller}
    
    \subsection{Insekten, Wirbelose, Wirbeltiere}
    
    \subsection{Rolle der Wasserpflanzen im Teich}
    
    \subsection{Benthal, Litoral}
    
    \subsection{Nahrungskette, Nahrungsnetz}
    
    \subsection{Stoffkreisläufe}


\section{Material und Methoden}
    
    \subsection{Aufspüren und Analyse der Lebewesen im Teich}
    
        % TODO Lu
    
        \begin{table}[h!]
        \centering
        \begin{tabular}{|c|} 
         \hline
         \\
         1 Zündholzschachtel \\
         1 Filzstift \\[1ex]
         \hline
        \end{tabular}
        \label{Tiere}
        \end{table}
        
    \subsection{Testen der Wasserqualität im Teich}
    
        \begin{table}[h!]
        \centering
        \begin{tabular}{|c|} 
         \hline
         \\
         1 Zündholzschachtel \\
         1 Filzstift \\[1ex]
         \hline
        \end{tabular}
        \label{Praktikum1}
        \end{table}

    

\section{Resultate}

    \subsection{Gefundene Lebewesen}
    
        % TODO Bas
    
        \subsubsection{Frühling}
        
        \subsubsection{Frühsommer}
        
    \subsection{Gemessene Wasserwerte}
        % TODO Lu


\section{Diskussion}

    \subsection{Nahrungsnetzwerke}
        
        % TODO Bas
    
        \subsubsection{Frühling}
        
        \subsubsection{Frühsommer}
        
    % \subsection{Planktonanalyse} potenziell
    
    \subsection{Wasseranalyse}
    
        % TODO Lu

\section{Danksagung}

\section{Literatur}

\end{document}
